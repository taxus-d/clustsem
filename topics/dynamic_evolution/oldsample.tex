\documentclass{beamer}

\usepackage{trclstools}
\input{trulang-luatex.ppk}

%\setmainfont{fonts/Exo2-Regular.otf}[
%BoldFont = {fonts/Lato-Semibold.ttf}
%]
%\setmonofont{CMU Typewriter Text}

\usetheme{Hannover}
\usecolortheme{default}
\setbeamercovered{transparent}
%\usefonttheme[onlymath]{serif}
%\setbeamerfont{title}{family=\fontspec{Lato}}


\usepackage{trmath}
\usepackage{trsym} 
% \usepackage{tikz}
\usepackage{siunitx}

\makeatletter
\define@key{beamerframe}{wide}[2ex]{%
    \def\beamer@cramped{\itemsep #1\topsep0.5pt\relax}}
\makeatother

%\usepackage{unicode-math}
%\setmathfont{STIXTwoMath.otf}
% Use XITS Math for    [     ]     (     )     {     }
% \setmathfont[range={"005B,"005D,"0028,"0029,"007B,"007D}]{XITS Math}
%\setmathfont[range={"007B,"007D}]{XITS Math}

\usepackage{wasysym}
\def\solmass{\mathfrak{M}_{\astrosun}}
\title{Формирование шаровых скоплений}
\author{Тихоненко И, 491}
%\institute{СПбГУ}
\date{18.10.2018}

\begin{document}
\begin{frame}
  \titlepage
\end{frame}

\begin{frame}{План}
  \tableofcontents
\end{frame}

\section{Когда?}

\begin{frame}[wide]{Время образования}
    Зависит от модели
    
    \begin{description}
    \item[$z\approx 6$] Реионизация Вселенной
    \item[$z\approx 2$] Пик звездообразования
    \end{description}
    
    \begin{block}{В нашей Галактике}
        \begin{description}
        \item[MP] 12.5 Gyr
        \item[MR] 11.5 Gyr
        \end{description}
    \end{block}
\end{frame}

\begin{frame}[wide]{Методы определения возраста}
    \begin{block}{Варианты}
    \begin{enumerate}
        \item Поворот ГП (Vandenberg et al, 2013)
        \item По охлаждению белых карликов (Hansen et al,2013; Garcia-Bero et al, 2014)
        \item Сопоставить металличность 
        соотношению масса-металличность для родных галактик на разных
        красных смещениях $\to$ эпоха. (E.g Spitler, 2012)
    \end{enumerate}
    \end{block}
\end{frame}

\section{Где?}
\begin{frame}[wide]{Механизмы\cite{royalproc}}
    
\begin{enumerate}
    \item \emph{in-situ} \\
        Маломассивные галактики\\
        <<внутренние>>, 
        высокометалличные (?)
    \item \emph{ex-situ} (аккреция)\\
        Массивные галактики \\
        <<внешние>>, 
        низкометалличные (?)
\end{enumerate}
Но отличить может быть сложно.

Работает и то, и другое
\end{frame}

\begin{frame}{Пример аккреции}

    \includegraphics[width=0.8\textwidth]{img/sgtstreams.jpg}

    $\approx 5$ ШС\cite{royalproc}
\end{frame}

\begin{frame}{Начальная масса системы ШС}
Сейчас $\mathfrak{M} \sim 1.2 \times 10^7 \solmass$, 

\[
\fder{N}{M} \propto (M + \Delta M) ^{-2} \, e^{-\frac{M +\Delta M}{M_c}}
\quad \text{\emph{evolved} Shelter}
\]
$\Delta M \sim \num{2.5e5} \solmass$, 
$M_c = \num{8e5}\solmass$

\vspace{1em}

\begin{block}{В итоге}
    $\sim \num{2.5e7}\solmass{}$
\end{block}

\begin{block}{Причины}
\begin{enumerate}
    \item Испарение в приливном потенциале
    \item Взаимодействие с ГМО
    \item Маленькие просто плохо образуются
\end{enumerate}
\end{block}

\end{frame}
\section{Как?}

\begin{frame}[wide]{Основные теории}
    \begin{enumerate}
        \item Особые условия в маломассивных тёмных гало в ранней Вселенной (реионизация)
        \item Обычное звездообразование\par
        \begin{enumerate}[a)]
            \item В слияниях галактик
            \item В диске
        \end{enumerate}
    \end{enumerate}
\end{frame}


\begin{frame}{Peebles, Dicke}

Основная идея~"--- джинсовская фрагментация первичных газовых облаков

\begin{block}{Причины}
    \begin{enumerate}
        \item Все ШС скопления похожи друг на друга
        \item Высокая плотность газа --- низкие скорости вращения
    \end{enumerate}
\end{block}
\end{frame}

\begin{frame}{Peebles, Dicke}
картиночка из статьи \cite{peebdicke}
    \includegraphics[width=0.6\linewidth,angle=90]{img/peebdspeckles.png}
\end{frame}

\begin{frame}{Fall, Rees}

\begin{block}{Проблемы}
    \begin{itemize}
        \item Скопления похожи на галактики рядом
        \item Нет межгалактических ШС
    \end{itemize}
\end{block}

\begin{block}{Выход}
Термальные нестабильности в газовых гало протогалактик
\end{block}

\end{frame}

\begin{frame}{Modern stuff}

\begin{itemize}
    \item Trenti et all, 2015
    \item Kimm et all, 2016
\end{itemize}

\vspace{1em}

Merging of two or more cooling atomic (H) halos at high redshift
($z > 6$).

\begin{quote}
    cosmological radiation-hydrodynamics simulations
\end{quote}
\end{frame}

\begin{frame}{Cлияния галактик}
    Li, Gnedin 2014
    
    \vspace{1em}
    
    \begin{enumerate}
        \item Формируются в гало богатых газом галактик
        \item Слияния гало формируют скопления
        \item Бимодальность металличности --- ранние слияние маломассивных гало и поздние массивных
    \end{enumerate}
    
\end{frame}

\begin{frame}{Звездообразование в диске}
    Kruijssen, 2015
    
    \vspace{1em}
    
    \begin{enumerate}
        \item Формируются в богатых газом областях в галактиках с $z>2$
        \item Должны как-то дожить до $z=0$ и не разрушится ударными
        волнами в газе (из-за приливов).
        \item Для этого приливными силами/слияниями транспортируются в гало
        и спокойно живут там.
    \end{enumerate}
\end{frame}

\section{Заключение}
\begin{frame}
\begin{center}
\Huge
    Всё сложно

\vspace{1em}

\Huge 
    Ничего не понятно
\end{center}
\end{frame}


\begin{frame}[allowframebreaks]{Список литературы}

\begin{thebibliography}{9}    
  \beamertemplatearticlebibitems
  \bibitem{royalproc}
    D.~Forbes et al.
    \newblock {\em Globular cluster formation and evolution in 
    the context of cosmological galaxy assembly: open questions}.
    \newblock arXiv:1801.05818v2 [astro-ph.GA] 31.01.2018
   \bibitem{sluggs}
    D.~Forbes et al.
    \newblock {\em The {SLUGGS} survey: inferring the formation epochs of metal-poor and metal-rich globular clusters}
    \newblock 2015MNRAS.452.1045F
  \bibitem{peebdicke}
    P.~Peebles and R.~Dicke
    \newblock {\em Origin of globular star clusters}.
    \newblock AJ, Vol 154, Dec 1968
  \bibitem{fallress}
    S.~Fall and M.~Rees
    \newblock {\em The theory for the origin of globular clusters}.
    \newblock 1985ApJ...298...18F
 \end{thebibliography}

\end{frame}


\end{document}
